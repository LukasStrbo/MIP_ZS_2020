\documentclass[10pt,oneside,slovak,a4paper]{article}

\usepackage[slovak]{babel}
%\usepackage[T1]{fontenc}
\usepackage[IL2]{fontenc}
\usepackage[utf8]{inputenc}
\usepackage{graphicx}
\usepackage{url} % príkaz \url na formátovanie URL
\usepackage{hyperref} % odkazy v texte budú aktívne (pri niektorých triedach dokumentov spôsobuje posun textu)

\usepackage{cite}
%\usepackage{times}

\pagestyle{headings}

\title{Aplikácie a riešenia dištančného vzdelávania a e-vzdelávania\thanks{Semestrálny projekt v predmete Metódy inžinierskej práce, ak. rok 2020/21, vedenie: Ing. Fedor Lehocki, PhD.}}

\author{Lukáš Štrbo\\[2pt]
	{\small Slovenská technická univerzita v Bratislave}\\
	{\small Fakulta informatiky a informačných technológií}\\
	{\small \texttt{xstrbol@stuba.sk}}
	}

\date{\small 29. október 2020}



\begin{document}

\maketitle

\begin{abstract}
E-vzdelávanie sa stáva stále viac populárnejšou metódou nadobúdania vedomostí. Mnohí ľudia ju preferujú najmä kvôli rýchlosti a efektívnosti získavania poznatkov.
Prostredníctvom internetu sa dokážeme vzdelávať pomocou rôznych aplikácií, webov, kurzov alebo aj diskusných fór.
S e-vzdelávaním ide ruka v ruke dištančné vzdelávanie, ktoré hlavne v ťažších časoch, ako je napríklad nemožnosť zúčastňovať sa prezenčnej výučby z dôvodu pandémie COVID-19, 
 je voľbou číslo jedna. Rozdiely v medzi e-vzdelávaním a dištančným vzdelávním si rozoberieme v kapitole \ref{rozdiely}. Cieľom tejto práce je sprehľadniť čitateľovi rôzne metódy dištančného vzdelávania. Rozoberieme si a porovnáme riešenia dištančného vzdelávania a ich priamu
 aplikáciu. Zameriame sa na výhody a nevýhody, ale aj ktoré softvéry alebo platformy sú lepšie pre dištančné vzdelávanie v praxi. 
 V rámci porovnávania sa zameriame aj na efektivitu a aplikáciu daných metód dištančného vzdelávania.
\end{abstract}



\section*{Úvod} %Nezobrazi sa cislovanie
\label{uvod}
%Uvod do problematiky
Vzdelávanie sa prostredníctvom počítača a webu sa stáva stále viac populárnejšou a častejšou metódou výučby či sa jedná o školy alebo o samoukov. V súvislosti aj s celosvetovou pandémiou 
 COVID-19 bola väčšina škôl nútená prejsť na dištančné vzdelávanie. Pod dištančným vzdelávaním rozuemieme aj e-learning. Tieto výrazy si rozoberieme v kapitole\ref{rozdiely}.  

\section{Rozdiel medzi dištančným vzdelávaním a e-vzdelávaním}
\label{rozdiely}
\subsection{Dištančné vzdelávanie}
Dištančné vzdelávanie zvyčajne zahŕňa situáciu, kde sú študenti oddelení od učiteľov na diaľku. Dištančné vzdelávanie zahŕňa poskytovanie systémov (elektronických alebo iných) na nadviazanie a udržiavanie komunikácie medzi učiteľmi a študentmi. Využíva určitú formu pedagogickej výmeny  medzi učiteľom a študentom na podporu učenia a hodnotenia.
Stará koncepcia dištančného vzdelávania bola spojená výlučne s tlačenými materiálmi, zatiaľ čo nová koncepcia dištančného vzdelávania zahŕňa doplnkový materiál používaný prostredníctvom netlačených médií, ako je rozhlas, televízia, počítače, notebooky, nahrané prednášky vo formáte videí, prostredníctvom projektorov, videokonferencií a interaktívnych stretnutí medzi študentmi.
Existujú 2 typy dištančného vzdelávania na základe interakcie študentov a to synchrónne a asynchrónne. Synchrónna metóda vyžaduje účasť študenta tvárou v tvár. Interakcia sa uskutočňuje v „reálnom čase“ a je bezprostredná. Asynchrónna nevyžaduje žiadnu účasť. Potreba, aby sa študenti a učitelia zhromaždili na stretnutí, je vylúčená a študenti si sami zvolia vlastný časový rámec pre interakciu. \cite{India}
\subsection{E-vzdelávanie}
E-learning je prirodzene vhodný na dištančné a flexibilné vzdelávanie, ale dá sa použiť aj v spojení s výučbou tvárou v tvár. V takom prípade sa bežne používa termín Blended learning. E-learning môže tiež odkazovať na vzdelávacie webové stránky, ako napríklad webové stránky ponúkajúce pracovné listy a interaktívne cvičenia pre deti. Tento výraz sa široko používa aj v obchodnom sektore, kde sa všeobecne vzťahuje na nákladovo efektívne online školenie. E-Learning je využitie technológií na podporu a zlepšenie výučby. So zameraním na používanie internetu v e-vzdelávaní sa objavili tri hlavné spôsoby použitia. Jedná sa o elektronickú technológiu na poskytovanie, podporu a zdokonaľovanie výučby a učenia sa.\cite{elearningDef}

\section{Typy dištančnej výučby}
%Uvod do typov distancnej vyucby

\subsection{Technology-based training (TBT)}
\subsection{Computer-based training (CBT)}
\subsection{Web-based training (WBT)}
\subsection{Instructor-led training (ILT)}
\subsection{Synchronous learning (SL)}
\subsection{Blended learning (BL)}


\section{Podporné systémy dištančnej vúčby}
\subsection{Komunikačné platformy}
%Webex, Meet, MS TEAMS, porovnanie , obluba u studentov, vlastnosti
\subsection{Edukačné platformy}
%Edukacne platformy - Online testovanie , Edupage, AIS, Kurzy online, Moodle, ...

\section{Aplikácia dištančnej výučby v praxi}
\subsection{Základné a stredné školy}
\subsection{Vysoké školy}
\subsection{Virtuálne univerzity}

\section{Efektivita a dopad dištančnej výučby}
\subsection{Efektívnosť}
\subsection{Výhody a nevýhody}


%\acknowledgement{Ak niekomu chcete poďakovať\ldots}



\bibliography{zdroje}
\bibliographystyle{plain}
\end{document}
